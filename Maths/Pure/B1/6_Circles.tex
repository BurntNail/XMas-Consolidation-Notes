Circles all follow a formula: \((x-a)^2 + (y-b)^2 = r^2\). \((a,b)\) is the centre of the circle, and \(r\) is the radius.
Questions might ask you to describe a circle given a formula - just rearrange till you can get to the formula.
When rearranging from the formula, be careful to preserve all solutions, as y is squared so you need to preserve negative values of y.

\section{Intersections between lines and circles}
Substitute into the circle equation. Make sure to check the discriminant or graph to check how many solutions exist.

\qs{Intersection between \(y=2x+1\) and \((x-3)^2 + (y+1)^2 = 64\)}{
	\begin{align*}
		(x-3)^2 + (2x+1+1)^2 &= 64 \\
		x^2 - 6x + 9 + 4x^2 + 8x + 4 &= 64 \\
		5x^2 + 2x - 51 &= 0 \\ \\
		2^2 - 4*5*-51 &> 0 \therefore \text{2 Intersections} \\ \\
		(5x+17) (x-3) &= 0 \\
		x &= 3, -3.4 \\
		y &= 2x + 1 \\
		y &= 7,-5.7 \\ \\
		&= (3,7) (-3.4, -5.8)
	\end{align*}
}