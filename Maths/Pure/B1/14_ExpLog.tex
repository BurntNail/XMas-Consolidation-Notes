\section{Logarithms}

\dfn{Logarithms}{
	\(a^x = y \therefore \log_ay=x\)
}
\begin{tabular}{cc}
	Exponential Fact & Logarithm Fact \\ \\
	\hline \\
	\(10^3 = 1000\) & \(\log_{10}1000 = 3\) \\ \\
	\(5^4 = 625\) & \(\log_5{625} = 4\) \\ \\
	\(36^{\frac{1}{2}} = 6\) & \(log_{36}6 = \frac{1}{2}\) \\ \\
	\(2^{-3} = \frac{1}{8}\) & \(log_2{\frac{1}{8}} = 3\)
\end{tabular}

\subsection{Log Laws}
\dfn{Log \& Index Laws}{
	\begin{tabular}{cc}
		Laws of Indices & Laws of Logs \\ \\
		\hline \\
		\(a^x * a^y = a^{x+y}\) & \(\log_ax + \log_ay = \log_amn\) \\ \\
		\(a^x \div a^y = a^{x-y}\) & \(\log_ax - \log_ay = \log_a\frac{x}{y}\) \\ \\
		\(\left( a^x \right) ^y = a^{xy}\) & \(\log_ax * \log_ay = \log_am^n\)
	\end{tabular}
}

\qs{\(\log_4\frac{x}{x-1} = \log_43 + log_42\)}{
	\begin{align*}
		\log_4\frac{x}{x-1} &= \log_46 \\
		\frac{x}{x-1} &= 6 \\
		x &= 6x - 6 \\
		\ldots
	\end{align*}
}
\qs{\(\log_74x = \log_7\frac{1}{x-6} + 1\)}{
	\begin{align*}
		\log_7\frac{4}{\frac{1}{x-6}} &= 1 \\
		\log_74x(x-6) &= 1 \\
		\log_74x^2-24x &= 1 \\
		4x^2-24x &= 7 \\
		\ldots
	\end{align*}
}
\qs{\(2^x=75\)}{
	\begin{align*}
		x &= \log_275 \\
		x &= 6.23
	\end{align*}
}
\qs{\(5^{2x} - 6\left( 5^x \right) = 0\)}{
	\begin{align*}
		\text{let } y &= 5^x \\
		y^2 - 6y - 7 &= 0 \\
		(y - 7)(y + 1) &= 0 \\
		\text{Logarithms always positive} \therefore y &= 7 \\
		5^x &= 7 \\
		x &= log_57 \\
		x &= 1.21 \\
	\end{align*}
}
\qs{\(3^{2x+1} = 2^{5x}\)}{
	\begin{align*}
		\log_e3^{2x+1} &= \log_e2^{5x} \\
		(2x+1)\log_e3 &= (5x)\log_e2 \\
		2(\log_e3)x + \log_e3 &= 5(log_e2)x \\
		\log_e3 &= 5(\log_e2)x - (2log_e3)x \\
		\log_e3 &= x(5log_e2 - 2loge_3) \\
		x &= \frac{log_e3}{5log_e2 - 2loge_3} \\
		x &= 0.866
	\end{align*}
}


\section{Exponentials}
\begin{plotter}
	\addplot [
		domain=-2:2,
		samples=100,
		color=black,
	]
	{2^x};
	\addlegendentry{\(2^x\)}
	\addplot [
		domain=-2:2,
		samples=100,
		color=black,
	]
	{2*2^x};
	\addlegendentry{\(2*2^x\)}
\end{plotter}

Graphs with anything to the power of \(x\) always have a y-intercept of 1, because anything to the power of 0 is equal to 1.


\subsection{Euler's Number - \(e\)}
\dfn{\(e\)}{
	\(e\) is defined as having the following characteristics:
	\begin{itemize}
		\ii \(y = e^x\), \(\differen = e^x\)
		\ii \(y = e^{kx}\), \(\differen = ke^{kx}\)
	\end{itemize}
}
\qs{\(y = Ae^{kx}\), and passes through \((0,6)\) and \(1,9\). Find \(A\) and \(k\)}{
	\begin{align*}
		6 &= Ae^0 \\
		A &= 6 \\ \\	
		9 &= 6e^k \\
		e^k &= \frac{9}{6} \\
		k &= \ln\frac{9}{6} \\
		k &= 0.41
	\end{align*}
}

\section{Modelling With Exponentials}
All Exponentials can be written with \(e\), which makes calculus much easier.

Given \(y = Ae^{kx}\), if \(k > 0\) we get exponential growth, and if \(k < 0\) then we get exponential decay.

% got up to just before 29/9 DJI before lunch