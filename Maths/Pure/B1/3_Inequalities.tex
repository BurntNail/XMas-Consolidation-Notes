\section{Basics}
Very similar to solving a normal equation with one caveat - every time you multiply by -1, you need to flip the sign.


\ex{\(5 - 3x \geq 21\)}{
	\begin{align*}
		5 - 3x &\geq 21 \\
		-3x &\geq 16 \\
		3x &\leq -16 \\
		x &\leq -\frac{16}{3}
	\end{align*}
}
\ex{\(17 + x \leq 32 + 3x \leq 21 + x\)}{
	\begin{alignat*}{2}
		17 + x &\leq 32 + 3x &&\leq 21 + x \\
		17 &\leq 32 + 2x &&\leq 21 \\
		-15 &\leq 2x &&\leq -11 \\
		-\frac{15}{2} &\leq x &&\leq -\frac{11}{2}
	\end{alignat*}
}

\section{Set Notation}
Whilst we can use \(\le\) and the other signs, we can also use set notation, which makes some easier, and is especially useful for quadratics.

\begin{align*}
	x > 4 &&\to&& \left\lbrace x: x < 4 \right\rbrace  &&\equiv&& x \in (4, \inf) \\
	-15 \leq x \leq -11 &&\to&& \left\lbrace x: -15 \leq x \leq -11 \right\rbrace  &&\equiv&& x \in [-15, -11]
\end{align*}

\section{Quadratics}
The difficulty comes with quadratics which have multiple x-solutions. We need to draw a graph, and then check which way around which we should give our answer - a single set, or a union of 2.
\ex{\(2x^2 + 5x - 3 < 0\)}{
	\begin{plotter}
		\addplot [
			domain=-5:5,
			samples=100,
			color=black,
		]
		{2*x*x + 5*x - 3};
		\addlegendentry{\(2x^2 + 5x - 3\)}
	\end{plotter}

	\begin{align*}
		2x^2 + 5x - 3 &< 0 \\
		(2x - 1)(x + 3) &< 0 \\
		-3 < x < \frac{1}{2} \\
		\left\lbrace x: -3 < x < \frac{1}{2} \right\rbrace 
	\end{align*}
}

\ex{\(x^2 > 10 - 3x\)}{
	\begin{plotter}
		\addplot [
			domain=-10:10,
			samples=100,
			color=black,
		]
		{x*x};
		\addlegendentry{\(x^2\)}
		\addplot [
			domain=-10:10,
			samples=100,
			color=blue,
		]
		{10 - 3*x};
		\addlegendentry{\(10-3x\)}
	\end{plotter}

	\begin{align*}
		x^2 + 3x - 10 &> 0 \\
		(x+5)(x-2) &> 0 \\
		\left\lbrace x: x < -5 \right\rbrace &\cup \left\lbrace x: x > 2 \right\rbrace 
	\end{align*}
}