\section{Graph Appearances}
\subsection{Transformations}

\dfn{Translation}{
	\begin{itemize}[]
		\ii The graph of \( f(x-a) \) is the graph of \( f(x) \) translated right by \(a\) units.
		\ii The graph of \( f(x)+b\) is the graph of \( f(x) \) translated upwards by \(b\) units.
	\end{itemize}
}
\dfn{Scaling}{
	\nt{Never say shrink: always say stretch by a factor \( e \) where \( |e| < 1 \)}
	\begin{itemize}[]
		\ii The graph of \(cf(x)\) is the graph of \(f(x)\) stretched vertically by a factor of \(c\).
		\ii The graph of \(f(dx)\) is the graph of \(f(x)\) stretched horizontally by a factor of \(d^{-1}\).
	\end{itemize}
}
\ex{Basic Graph Transformations}{
	\begin{plotter}
		\addplot [
		domain=-5:5,
		samples=100,
		color=black,
		]
		{x^2};
		\addlegendentry{\(x^2\)}
	\end{plotter}
	
	\hbox{
		\begin{plotter}
			\addplot [
			domain=-5:5,
			samples=100,
			color=red,
			]
			{(x-2)^2};
			\addlegendentry{\((x-2)^2\)}
			
			\addplot [
			domain=-5:5,
			samples=100,
			color=orange,
			]
			{(x)^2+10};
			\addlegendentry{\((x)^2+10\)}
		\end{plotter}
		
		\begin{plotter}
			\addplot [
			domain=-5:5,
			samples=100,
			color=blue,
			]
			{3*(x^2)};
			\addlegendentry{\(3(x^2)\)}
			
			\addplot [
			domain=-5:5,
			samples=100,
			color=purple,
			]
			{(x/2)^2};
			\addlegendentry{\((0.5x)^2\)}
		\end{plotter}
	}
}

\subsection{Combining Transformations}

\qs{Combining Transformations}{
	\[
	y = f(-2x)
	\]
	
	This is obtained from \(f(x)\) by doing the following:
	\begin{enumerate}
		\item Flip horizontally.
		\item Stretch horizontally by a factor of \(0.5\).
	\end{enumerate}
}
\qs{Combining Transformations}{
	\[
	y = cf(\frac{1}{a} * (x-b)) + d
	\]
	
	This is obtained from \(f(x)\) by doing the following:
	\begin{enumerate}
		\item Shift to the right \(b\) units.
		\item Stretch horizontally by a factor of \(a\).
		\item Stretch vertically by a factor of \(c\).
		\item Shift upwards by \(d\) units.
	\end{enumerate}
}


\pagebreak
\subsection{Graph Shapes}
\ex{Basic Power Graphs}{
	\hbox {
		\plotbasic[blue]{x}{\(x\)}
		\plotbasic[red]{x^2}{\(x^2\)}
	}
	\hbox {
		\plotbasic[blue]{x^3}{\(x^3\)}
		\plotbasic[red]{x^4}{\(x^4\)}
	}
	\hbox {
		\plotbasic[blue]{1/x}{\(x^{-1}\)}
		\plotbasic[red]{-1/x}{\({-x}^{-1}\)}
	}
}

These are all of the basic graph shapes, and can be transformed just like \(y = x^2\) above.

\pagebreak
\section{Solving Using Graphs}
Find one or more \(y =\ldots \) equation, plot it, find the x position of any intercepts. If only one equation, find intersections with \(y = 0\).

\qs{Solving \(5 = 6x + 8\)}{
	\begin{align}
		5  & = 6x + 8 \\
		-3 & = 6x     \\
	\end{align}
	
	\ex{Using Algebra}{
		\begin{align}
			6x & = -3           \\
			x  & = \frac{6}{-3} \\
			x  & = -\frac{1}{2}
		\end{align}
	}
	
	\ex{Using a Graph}{
		\[ y = 6x + 3  \]
		\plotbasic{6*x+3}{\(6x + 3 \)}
		Intersects at \(-\frac{1}{2}\).
	}
	
	Whilst this might seem less useful for basic equations, this can become much more useful for more complicated questions like below.
}

\qs{Finding the intersection of  \(y = 3x^2 -2x - 21\) and \(y = (x-3)(x+3)\)}{
	\ex{Using Algebra}{
		\begin{enumerate}
			\ii Set equal to each other
			\ii Simplify
			\ii Check how many roots exist using the discriminant
			\ii Work out all roots (possibly using factor theorem which can take a while)
		\end{enumerate}
		\begin{align}
			3x^2 -2x - 21  & = (x-3)(x+3) \\
			3x^2 -2x - 21  & = x^2 - 9    \\
			2x^2 -2x - 12  & = 0          \\
			x^2 - x - 6    & = 0          \\
			(x - 3)(x + 2) & = 0          \\
			x              & = 3, -2
		\end{align}
	}
	
	\ex{Using a Graph}{
		\begin{enumerate}
			\ii Plot
			\ii Check intersections.
		\end{enumerate}
		
		\begin{plotter}
			\addplot [
			domain=-7.5:7.5,
			samples=100,
			color=black,
			]
			{3*x^2 - 2*x - 21};
			\addlegendentry{\(3x^2 -2x -21\)}
			
			\addplot [
			domain=-7.5:7.5,
			samples=100,
			color=black,
			]
			{(x-3)*(x+3)};
			\addlegendentry{\((x+3)(x-3)\)}
		\end{plotter}
		
		Intersects at \(x = -2, 3\).
	}
}
