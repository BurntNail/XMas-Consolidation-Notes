\dfn{\(a\) and \(d\) notation} {
	\begin{tabular}{rl}
		Symbol  & Meaning \\
		\hline
		\(U_n\) & \(n\)\textsuperscript{th} term \\
		\(S_n\) & Sum of terms up to and including \(U_n\) \\
		\(a\) & First Term (AKA \(U_1\)) \\
		\(d\) & Common Difference \\
		\(r\) & Common Factor \\
	\end{tabular}
}
\dfn{Sigma Notation}{
	\[ \sum_{r=1}^{10} U_r = U_1 + U_2 + \ldots + U_{10} = S_{10} \]
}

\section{Arithmetic Series}
\subsection{\(U_n\)}
\dfn{Term Formula}{
	\[U_n = a + (n-1)d\]
}

\qs{\(U_5 = 17\), \(U_4 = 3\), Find \(U_n\)}{
	\begin{align*}
		17 &= a + 4d \\
		3 &= a + 8d \\ \\
		& \text{Solve Simultaneous Equations} \\
		a &= 31 \\
		d &= -\frac{7}{2} \\ \\
		U_n &= 31 +  \left( n-1 \right)  \left( -\frac{7}{2} \right) \\
		&= -\frac{7}{2}n + 34\frac{1}{2}
	\end{align*}
}

\subsection{\(S_n\)}
\dfn{Sum Formula}{
	\begin{align*}
		S_n &= \frac{n}{2}\left( \text{First} - \text{Last} \right) \\
		& \therefore \\
		S_n &= \frac{n}{2}\left( 2a + (n-1)d \right)
	\end{align*}
}

\qs{\( \sum_{n=1}^{20}5n-3 \)}{
	\begin{align*}
		\sum_{n=1}^{20}5n-3 &= \frac{n}{2}\left( 2a + (n-1)d \right) \\
		&= 10(4+19*5) \\
		&= 990
	\end{align*}
}


\section{Geometric Sequences}
\subsection{\(U_n\)}
\dfn{Term Formula}{\[U_n = ar^{n-1}\]}

\qs{Given terms \(3, x, x+6\), find \(U_{10}\)}{
	\begin{align*}
		\frac{x}{3} &= \frac{x+6}{x} \\
		x^2 &= 3x + 18 \\
		0 &= x^2 - 3x - 18 \\
		0 &= (x - 6)(x+3) \\
		x &= 6,-3 \\
	\end{align*}

	\begin{multicols}{2}
	\noindent
	\begin{align*}
		x &= 6 \\
		U &= 3,6,12 \\
		(a,r) &= (3,2) \\
		U_n &= 3*2^{n-1} \\
		U_{10} &= 3*2^9 \\
		&= 136
	\end{align*}
	\columnbreak
	\begin{align*}
		x &= -3 \\
		U &= 3,-3,3 \\
		(a,r) &= (3,-1) \\
		U_n &= 3*-1^{n-1} \\
		U_{10} &= 3*2^9 \\
		&= -3
	\end{align*}
	\end{multicols}

	\[\therefore x = -3, 136\]
}
\qs{Given terms \(2,6,18\), find \(n\) where \(U_n > 500,000\)}{
	\begin{align*}
		(a,r) &= (2,3) \\
		U_n &= 2 * 3^{n-1} \\
		2 * 3^{n-1} &> 500000 \\
		3^{n-1} &> 250000 \\
		n &> \log(250000)  1\\
		n &> 12.3 \\
		n &= 13 \\
	\end{align*}
}



\subsection{\(S_n\)}
\dfn{Sum Formula}{\[ S_n = \frac{a\left( 1-r^n \right) }{1-r} \]}

\qs{\(1024 - 512 + 256 - 128 + \ldots + 1 = x\), find \(x\)}{
	\begin{align*}
		(a, r) &= (1024, -\frac{1}{2}) \\
		n &=\log_2(1024) + 1 \\
		&= 11 \\
		S_n &= a\frac{1-r^n}{1-4} \\
		&= 1024 \frac{1+\frac{1}{2}^{11}}{1+\frac{1}{2}} \\
		&= 683
	\end{align*}
}

\subsection{Geometric Sequences to Infinity}
\dfn{Sum to Infinity}{
	If \(U_n = ar^n\), and \(\left| r \right|  < 1\), then \(S_\infty = \frac{a}{1-r} \).
}
\dfn{Convergent vs Divergent}{
	Given \(U_n = a + (n-1)d\)
	\begin{alignat*}{2}
	\limit{U_n}{\infty} & d > 0 & \text{Tends to } \infty\\
				   		  & d < 0 & \text{Tends to } -\infty\\
	\limit{S_n}{\infty}    & d > 0 & \text{Tends to } \infty\\
						  & d < 0 & \text{Tends to } -\infty\\
	\end{alignat*}
}


\section{Recurrence Relations}
\dfn{Recurrence Relations}{
	A recurrence relation is a term-to-term rule for a sequence denoted by \(U_{n+1} = \func[U_n]{f}\), with a value given for \(U_1\)
}