\chapter{Forces In Equilibrium}
% vectors/scalars balanced_forces moments stability

\section{Vectors}
\subsection{Examples}
\begin{tabular}{ll}
	\textbf{Scalar Quantities} & \textbf{Vector Quantities} \\
	\hline & \\
	Energy                     & Force                      \\
	Charge                     & Momentum                   \\
	Speed                      & Velocity                   \\
	Distance                   & Displacement               \\
	Mass                       & Weight                     \\
	Temperature                & Acceleration               \\
	Time                       & Jerk                       \\
	Density                    & Snap                       \\
	Pressure                   & Crackle                    \\
	& Pop                       
\end{tabular}

\subsection{Combining Vectors}
We can either use lots of Trigonometry, or draw out the diagram to scale, top-to-tail and record the line from end to start, and that is the resultant vector.

If we can't use a diagram, and we are given a complicated example, turn each vector to \(x\) and \(y\) components using trig and add them together.


\section{Motion}
\subsection{Motion Graphs}
\begin{tabular}{lll}
	\textbf{Y-Axis}       & \textbf{Gradient Shows} & \textbf{Area under Shows} \\
	\textbf{Displacement} & Velocity                & -                         \\
	\textbf{Velocity}     & Acceleration            & Displacement             
\end{tabular}

\subsection{SUVAT Equations}
\dfn{SUVAT}{
	\begin{align}
		S &= \text{Displacement} \left[ m \right] \\
		U &= \text{Initial Velocity} \left[  ms^{-1} \right] \\
		V &= \text{Final Velocity} \left[ ms^{-1} \right] \\
		A &= \text{Acceleration} \left[ ms^{-2} \right]  \\
		T &= \text{Time} \left[ s \right] 
	\end{align}
}
\dfn{SUVAT Equations \label{Eq_SUVAT}} {
	Technically, there are 3 SUVAT equations but these have been rearranged so that each is missing one 4, except for u which if missing \( \equiv 0 \)
	\begin{align}
		v &= u + at \label{Eq_SUVAT_NoS}  \\
		s &= \frac{u + v}{2}t \label{Eq_SUVAT_NoA} \\
		s &= ut + \frac{at^2}{2} \label{Eq_SUVAT_NoV} \\
		v^2 &= u^2 + 2as \label{Eq_SUVAT_NoT}
	\end{align}
	\begin{itemize}
		\ii \eqref{Eq_SUVAT_NoS} is missing \(s\)
		\ii \eqref{Eq_SUVAT_NoA} is missing \(a\)
		\ii \eqref{Eq_SUVAT_NoV} is missing \(v\)
		\ii \eqref{Eq_SUVAT_NoT} is missing \(t\)
	\end{itemize}
}

\subsection{Projectile Motion}
Here, we have both \(x\) and \(y\) to consider, but the only force acting on the object is weight. This means that \(a_x = 0\) and \(a_y = -9.81\).

An object undergoing projectile motion will follow a parametric path, for example:
\begin{itemize}
	\ii A kicked ball
	\ii A fired bullet
	\ii A dropped phone
\end{itemize}

\subsection{SUVAT Tips}
\begin{itemize}
	\ii Choose a consistent start and end point
	\ii Set clear positive directions
	\ii If drag doesn't apply, \(a_x = 0\)
	\ii Trajectories are symmetrical around the maximum point - this can help for finding the peak of a kicked ball, for example
\end{itemize}




\chapter{Forces}
There are 4 fundamental forces:
\begin{enumerate}
	\ii Gravitational
	\ii EM
	\ii Weak Nuclear
	\ii Strong Nuclear
\end{enumerate}

All forces we think are their own forces come from these 4. For example Weight comes from Gravity (\(W = mg\)). All of the following come from EM:
\begin{itemize}
	\ii Normal
	\ii Friction
	\ii Drag
	\ii Upthrust
	\ii Thrust
	\ii Tension
	\ii Lift
\end{itemize}

To represent the forces acting on a body, we can use Free-Body Diagrams, where we draw where the forces on an object act from.

\dfn{Newton's \nth{3} Law}{
	If Object A exerts a force on object B, then object B will exert an equal and opposite force on object A.
}

\section{Quick-Fire Notes}
\subsection{Equations}
\begin{tabular}{lr}
	\textbf{Subject} & \textbf{Equation} \\
	\hline & \\
	\text{Work Done} & \( W = fd \) \\ & \\
	\text{Kinetic Energy} & \( E_K = \frac{1}{2}mv^2 \) \\ & \\
	\text{Gravitational Potential Energy} & \( E_P = mgh \) \\ & \\
	\text{Power} & \( P = \frac{W}{t} = Fv \) \\ & \\
	\text{Efficiency} & \( E = \frac{\text{Useful Out}}{\text{Total In}} \) \\ & \\
	\text{Two-Support} & \( F_a = \frac{W D_b}{D} \quad F_b = \frac{W D_a}{D} \) \\ & \\
	\text{Tilting} & \( Fd > \frac{Wb}{2} \)
\end{tabular}

\subsection{Work \& Energy}
\begin{itemize}
	\ii Energy is measured in Joules. \(1J\) is the energy required to raise a \(1N\) weight \(1m\) vertically.
	\ii Energy can neither be created nor destroyed.
	\ii Work is done on an object when a force acting on it makes it move.
	\ii A force-distance graph shows the forces acting on an object.
	\ii The area under a force-distance graph shows the work done.
\end{itemize}

\subsection{Kinetic \& Potential}
\dfn{Kinetic Energy}{The Energy of an object due to its motion.}
\dfn{Potential Energy}{The energy of an object due to its position.}

\subsection{Power \& Energy}
\dfn{Power}{The rate of transfer of energy.}

\subsection{Centre of Mass}
\dfn{Centre of Mass}{The point through which a single force on the body has no turning effect.}


\section{Stability}
If a body in stable equilibrium is displaced and then released, it returns to its equilibrium position. This is because the CoM is directly below the point of support when at rest., so the support force and weight are in equilibrium. Therefore, when displaced the weight force tries to go back.
If a body is in unstable equilibrium, then all is reversed: the support is above the weight, and when displaced the object leaves the equilibrium.
An object will topple if the line of action from its weight passes over the pivot.



\chapter{Momentum}
\[P = mv\]
\begin{align}
	P &= \text{Momentum} \left[ kgms^{-1} \right] \text{or} \left[ Ns \right] \\
	m &= \text{Mass} \left[ kg \right] \\
	v &= \text{Velocity} \left[ ms^{-1} \right] 
\end{align}

Momentum is a vector.

\section{Conservation of Momentum}
\dfn{Conservation of Momentum}{
	The total momentum before a collision is equal to the total momentum after a collision, provided that no external forces are acting.
	\[ \sum{P_1} \equiv \sum{P_2} \therefore P_1 = P_2 \]
}

\section{Forces from Momentum}
\[F = \frac{\Delta P}{t}\]
\begin{align}
	F &= \text{Force} \left[ N \right] \\
	P &= \text{Momentum} \left[ kgms^{-1} \right] \text{or} \left[ Ns \right] \\
	t &= \text{Time} \left[ s \right] 
\end{align}

A large force will result if the \(\Delta P\) is large or if the time is short (ie if the change occurs very quickly).

\section{(In)Elastic Equations}
\dfn{Elastic Collision}{
	An Elastic Collision is one where Kinetic Energy is conserved.
}
\dfn{Inelastic Collision}{
	An Inelastic Collision is one where Kinetic Energy is \textbf{not} conserved. 
	
	A sign of this is objects sticking together and moving together post-collision or any deformations.
}