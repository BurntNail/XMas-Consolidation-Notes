\chapter{Uncertainty}
This refers to how precise a given measurement is, with an error amount. Usually, this is \( \pm 1 \) of the smallest digit.
\ex{Reading \eqref{Reading} Uncertainties}{
	\begin{itemize}
		\ii \(1.537g \to 1.537 \pm 0.001g\)
		\ii \(1.2g \to 1.2 \pm 0.1g\)
	\end{itemize}
}

\dfn{Uncertainty of Multiple Measurements}{
	\begin{tabular}{ll}
		\textbf{Example Equation} & \textbf{Uncertainty Equation} \\
		\hline & \\
		\(A + B\) or \(A - B\)	  & \( A_{\left| \text{UC} \right|} + B_{\left| \text{UC} \right|} \) \\
		\(A * B\) or \(A / B\)	  & \( A_{\% \text{UC}} + B_{\% \text{UC}} \) \\
		\(A^n\)					  & \( A_{\% \text{UC}} * n \)
	\end{tabular}
}



\section{Uncertainty in Graphs}
To get the uncertainty in a Graph, we need to firstly draw the lines of best and worst acceptable fit. Then we can calculate the percentage difference between the gradients for the \textbf{gradient uncertainty}, as well as the percentage difference in the y intercepts to get the \textbf{y intercept uncertainty}.

\ex{Gradient Uncertainty}{
	\textbf{Line of Best Fit:}
	\begin{tabular}{cc}
		\(x\)     & \(y\) \\
		\hline & \\
		\(1.7\)   & \(0.41\) \\
		\(1.975\) & \(0.85\) \\
	\end{tabular}
	\[ m = \frac{0.85 - 0.41}{1.975-1.7} = 1.6 \]
	
	\textbf{Line of Worst Acceptable Fit:}
	\begin{tabular}{cc}
		\(x\)     & \(y\) \\
		\hline & \\
		\(1.7\)   & \(0.39\) \\
		\(1.975\) & \(0.86\) \\
	\end{tabular}
	\[ m = \frac{0.86 - 0.39}{1.975-1.7} = 1.7 \ldots \]
	
	\textbf{Gradient Uncertainty: }
	\[ \%m = \frac{|m_w - m_b|}{m_b} = \frac{|1.7 - 1.6|}{1.6} = \frac{0.1}{1.6} = 6.3\% \]
}

\ex{Y-Intercept Uncertainty}{
	\textbf{Line of Best Fit:}
	\begin{tabular}{cc}
		\(x\)     & \(y\) \\
		\hline & \\
		\(1.7\)   & \(0.41\) \\
		\(1.975\) & \(0.85\) \\
	\end{tabular}
	\[ c = 0.41 - (1.6 * 1.7) = -2.3 \]
	
	\textbf{Line of Worst Acceptable Fit:}
	\begin{tabular}{cc}
		\(x\)     & \(y\) \\
		\hline & \\
		\(1.7\)   & \(0.39\) \\
		\(1.975\) & \(0.86\) \\
	\end{tabular}
	\[ c = 0.39 - (1.709 * 1.7) = -2.5 \]
	
	\textbf{Y-Intercept Uncertainty: }
	\[ \%c = \frac{|c_w - c_b|}{|c_b|} = \frac{|-2.5 - -2.3|}{|-2.3|} = \frac{0.2}{2.3} = 8.7\% \]
}


\chapter{SI}

\section{Powers of \(10\)}
Most Physicists work in powers of \(10\), going up and down by \(10^3\), and here are the SI Prefixes: \\
\begin{tabular}{lcc}
	Prefix & Power of \(10\) & Symbol \\
	\hline & \\
	Tera-	 & \(12\)			 & T \\
	Giga-	 & \(9\)			  & G \\
	Mega-	 & \(6\)			  & M \\
	Kilo-	 & \(3\)			  & k \\
	Deci-	 & \(-1\)			 & d \\
	Centi-	 & \(-2\)			 & c \\
	Mili-	 & \(-3\)			 & m \\
	Micro-	 & \(-6\)			 & \(\mu\) \\
	Nano-	 & \(-9\)			 & n \\
\end{tabular}


\section{SI Base Units}
The SI decided that the following are Base Units - they are indivisible, unlike other units like \(Pa\) which are combinations of other units (\(1 Pa \equiv 1 Nm^{-2}\), see \eqref{DimensionalAnalysis} for more) \\

\begin{tabular}{clc}
	Unit     & Measures                  & Repr  \\
	\hline & \\
	Ampere   & Electric Current          & A     \\
	Candela  & Luminous Intensity        & cd    \\
	Kelvin   & Thermodynamic Temperature & K     \\
	Kilogram & Mass                      & kg    \\
	Metre    & Length                    & m     \\
	Mole     & Amount of Substance       & mol   \\
	Second   & Time                      & s    
\end{tabular}


\chapter{Dimensional Analysis \label{DimensionalAnalysis}}

\section{Measurements \& Readings}
\dfn{Reading \label{Reading}}{The Value of an instrument.}
\dfn{Measurement}{The difference between 2 readings.}

\section{Dimensional Analysis}

Dimensional Analysis is combinations of units.

\ex{\(F = ma\), work out the units of \(F\)}{
	\begin{align*}
	F &= ma \\
	\left[ N \right] &= \left[ kg \right] \left[ ms^{-2} \right] \\
	N &= kgms^{-2}
	\end{align*}
}
\ex{\(E_K = \frac{1}{2}mv^2\), work out the units of \(E_K\)}{
	\begin{align*}
	E_K &= \frac{1}{2}mv^2 \\
	\left[ J \right] &= \left[ kg \right] \left[ ms^{-1} \right]^2 \\
	J &= kgm^2s^{-2}
	\end{align*}
}
\ex{\(F = \frac{Gm_1m_2}{r^2}\), work out the units of \(G\)}{
	\begin{align*}
	F &= \frac{Gm_1m_2}{r^2} \\
	G &= \frac{Fr^2}{m_1m_2} \\
	\left[  G \right] &= \left[ N \right] \left[ m \right]^2\left[ kg \right]^{-2} \\
	G &= m^3kg^{-1}s^{-2}
	\end{align*}
}